% @Author: bo
% @Date:   2016-05-30 08:28:29
% @Last Modified by:   bo
% @Last Modified time: 2016-06-06 10:48:30

\chapter{总结与展望}
\label{cha:conclusion}

\section{本文总结}
\label{sec:conclusion}

本文研究了多方法多模态信息融合策略以及深度学习在多模态信息融合中的应用,具体的工作有:
\begin{enumerate}
  \item 首先,我们利用传统的分层匹配追踪,以及随机权值卷积递归神经网络来进行颜色和深度信息的提取和融合,并探究不同模态的信息之间的互补性。然后我们在这两种方法的基础上提出了一种决策层的信息融合策略,并探究不同方法之间的互补性。实验表明,不仅仅不同模态的信息之间具有互补性,不同信息提取方法之间也具有很强的互补性。将不同模态的信息或不同信息提取方法进行融合都可以使得我们对目标物体特征的感知变得更加全面,可以获得更好的识别效果。大量已有工作都是仅限于利用多传感器或不同模态之间的融合,而我们引入的不同方法之间融合的思路,可以进一步利用数据的互补性,弥补各个方法的不足。
  \item 其次,本文在传统信息提取方法之外还引入了深度学习方法。即使用八层的大型卷积神经网络在颜色信息和深度信息中分别进行深度学习,然后将相应的特征层信息提取出来,并进行模态之间的特征层信息融合,最终给出识别结果。此外,本文还提出了一种将深度信息转换为图片信息的正规化方法。这种方法对于深度信息的不同部分采用不同的正规化尺度,从而实现了既保留目标物体与背景之间的间隔,同时又保留目标物体之内形状变化信息的目的。实验证明,所提的深度信息正规化方法可以提升深度学习的识别准确率。
  \item 同时,为了解决深度学习中数据量不足的问题,我们使用了预先在海量图片数据集中训练过的模型来进行信息提取。而且由于已有模型是在颜色信息数据集中训练的,我们使用已有深度信息对其进行了微调,使它可以更好地适应深度信息的提取。试验证明,使用RGB图片预先训练的深度学习模型可以较好地识别颜色和深度信息。最后,我们基于以上研究提出了综合分层匹配追踪、随机权值递归神经网络以及深度学习三种方法的信息融合识别模型。实验证明,我们的融合模型可以提升识别的准确率和稳定性,并且降低了深度学习对于数据量的依赖。
\end{enumerate}

\section{未来工作展望}
\label{sec:future}

本文研究了多方法多模态信息融合策略,对不同层次的多种信息融合策略进行了较深入的探讨和实验,提高了分类的准确率和稳定性。不过仍有一些可以进一步完善的方向。具体有:
\begin{enumerate}
\item 进一步探究多模态信息融合策略。
在本文的探讨与实验中,我们完善了多模态信息融合方面的策略与方法。不过在实验中我们发现,多模态信息融合的潜力还有待开发,应用空间还很广阔。因此,进一步开发多模态信息融合的潜力,以便在各类机器人应用中发挥更大的效果是颇有意义的。
\item 探究深度学习在其他模态识别等任务中的应用。
我们知道传统深度学习需要的数据量是非常可观的,因而只在图像识别、分类、分割等应用中获得过较好的结果。但在本文中我们成功的将深度学习应用到了深度信息当中,尽管我们使用深度信息的数据规模并不是很大。这就为我们将深度学习的方法和模型应用到其他模态和领域开辟了道路。在本文中我们将只是将深度信息正规化为图片信息就可以使用预训练的模型取得较好的分类结果,下一步我们可以将其他模态的信息也编码为图片信息,并使用预训练的模型进行微调和分类。
\item 探究更多模态的融合。
本文只是将颜色信息和深度信息结合了起来,就获得了识别准确率和稳定性上的显著提升。然而现实中的信息模态远远不止这两种,而且有些模态与RGB-D的信息之间具有天然的更大的互补性。例如触觉信息和滑觉信息可以告诉我们目标物体的材质力学信息和摩擦力信息等等。而现实中的机器人操作往往是在错综复杂的环境中进行的,在机器人的感知和操作中考虑更多模态的融合,将会使机器人获得对环境的更加全面的认知,并作出更加智能的决策。
\end{enumerate}
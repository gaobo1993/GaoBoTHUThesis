\chapter{背景知识}
\label{cha:background}

本章主要介绍一些与本文研究相关的背景知识。

\section{多模态信息融合}
\label{sec:sec21}

\subsection{多模态信息融合的概念}
信息,根据香农(Shannon)的定义,是“用来消除事物不确定性的数据”,并且使用熵来量化描述信息的质量~\inlinecite{shannon2001mathematical} 。控制论奠基人维纳(Norbert Wiener)认为信息是“人在与外部世界交互、适应并反作用于外部世界时,同外部世界交换的内容”~\inlinecite{shannon1949mathematical} 。以上可以看作是对于信息的经典的定义。不过现代科学技术日新月异,现在将使用信息的主语限定于人已经不是十分恰当了,因为计算机和机器人也可以使用信息做出判断,维纳研究的控制论就是计算机利用信息的例子~\inlinecite{wiener1961cybernetics} 。狭义的多模态信息指的是使用不同种类的传感器收集到的信息,这其中的重点是“多种传感器”,它相对于多传感器信息,强调因为传感器和信息种类的不同而产生了不同来源、不同结构的多模态信息~\inlinecite{raman1998multimodal} 。而广义的多模态信息则主要强调信息结构上的异构性,因为异构性而产生的互补性~\inlinecite{mansoorizadeh2010multimodal} 。

对于多模态信息融合的定义,我们选用Joint Directors of Laboratories(JDL)的定义:对多种信息在多个层次上、使用多种方法的联合处理过程,以便实现多种信息的互相组合、互相关联和联合利用~\inlinecite{white1991data} 。使用多模态信息融合的意义是更好地利用不同模态的信息之间的互补性,以便提供相辅相成的综合认知和理解,提高整体决策的可靠性和鲁棒性~\inlinecite{atrey2010multimodal} 。以及对多个或多种传感器收集的信息进行联合利用,获得比单个或单一种类的传感器更加准确的结果~\inlinecite{waltz1990multisensor} 。

\subsection{多模态信息融合的层次}
实际操作中我们对于信息的处理往往都是分层进行的~\inlinecite{shiffrin1977controlled} ,而对于多模态信息的处理,往往可以分为三个层次~\inlinecite{xiong2002multi} ,分别是:
\begin{enumerate}
\item 数据层
\item 特征层
\item 决策层
\end{enumerate}
下面逐一进行说明。

数据层。对于不同传感器收集到的原始信息直接进行处理,称为数据层的操作。在这一层次进行处理和融合的优点有:包含第一手原始数据,因为没有经过信息提取,也就没有信息损耗,可以说含有的信息量是最大的。缺点包括:数据量庞大,数据维数高,表征效率低,因而给信息提取造成了方法上和效率上的困难;同时受噪声影响比较大。

特征层。对底层数据进行初步处理,并且根据需求提取出来的特征,称为特征层。特征层的特征基于对底层数据的提取和抽象,可能已经包含一定的经过提炼的模式信息,比如使用单层CNN提取的信息会包含物体的边、角、形状等模式~\inlinecite{caudill1994understanding} 。在这一层次进行处理和融合的优点有:经过提取的信息往往已经对于原始数据中某些规律有所把握,具有更适合于某种特定任务(比如分类、分割)的结构,很大程度上消除了噪声的影响,数据维度适中,因而在此层面上的处理效率较高;同时和决策层相比,特征层信息的灵活性更大。缺点包括:在提取过程中可能造成一定的信息损失,且没有数据层的操作灵活。

决策层。对数据层或特征层的信息加以自动化或半自动化的分析提炼,得出具有一定语义的结果,即为决策层信息~\inlinecite{prabhakar2002decision} 。决策层基于特政策和数据层,经过进一步的提炼,已经得出了可以被人理解和使用的语义信息。在这一层次进行处理和融合的优点有:数据维度已被高度压缩,在这一层的处理和判断时间及空间复杂度很低,抗噪声干扰的能力最强;并且由于具有语义信息,因而可以被人直接理解和利用。而缺点则包括:由于信息高度精炼,可能有较多的信息损失,且在这一层实施融合策略的灵活性很低,往往只可以使用投票~\inlinecite{chatzis1999multimodal} 、专家系统~\inlinecite{clancey1983epistemology} 等高层方法。


% \subsection{关键问题}


% \section{信息提取方法}
% \label{sec:sec22}

% \subsection{Bag of Words}

% \subsection{分层匹配追踪}

% \subsection{卷积递归神经网络}

% \section{深度学习与ILSVRC}
% \label{sec:sec23}
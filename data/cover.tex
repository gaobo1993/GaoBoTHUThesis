\thusetup{
  %******************************
  % 注意:
  %   1. 配置里面不要出现空行
  %   2. 不需要的配置信息可以删除
  %******************************
  %
  %=====
  % 秘级
  %=====
  secretlevel={秘密},
  secretyear={10},
  %
  %=========
  % 中文信息
  %=========
  ctitle={深度学习与多模态信息融合策略探究},
  cdegree={工学学士},
  cdepartment={计算机科学与技术系},
  cmajor={计算机科学与技术},
  cauthor={高博},
  csupervisor={李洪波 助理研究员},
  % cassosupervisor={某某某教授}, % 副指导老师
  % ccosupervisor={某某某教授}, % 联合指导老师
  % 日期自动使用当前时间,若需指定按如下方式修改:
  % cdate={超新星纪元},
  %
  % 博士后专有部分
  cfirstdiscipline={计算机科学与技术},
  cseconddiscipline={系统结构},
  postdoctordate={2009年7月——2011年7月},
  id={编号}, % 可以留空: id={},
  udc={UDC}, % 可以留空
  catalognumber={分类号}, % 可以留空
  %
  %=========
  % 英文信息
  %=========
  etitle={Deep Learning and Multimodel Information Fusion Strategies},
  % 这块比较复杂,需要分情况讨论:
  % 1. 学术型硕士
  %    edegree:必须为Master of Arts或Master of Science(注意大小写)
  %             “哲学、文学、历史学、法学、教育学、艺术学门类,公共管理学科
  %              填写Master of Arts,其它填写Master of Science”
  %    emajor:“获得一级学科授权的学科填写一级学科名称,其它填写二级学科名称”
  % 2. 专业型硕士
  %    edegree:“填写专业学位英文名称全称”
  %    emajor:“工程硕士填写工程领域,其它专业学位不填写此项”
  % 3. 学术型博士
  %    edegree:Doctor of Philosophy(注意大小写)
  %    emajor:“获得一级学科授权的学科填写一级学科名称,其它填写二级学科名称”
  % 4. 专业型博士
  %    edegree:“填写专业学位英文名称全称”
  %    emajor:不填写此项
  edegree={Bachelor of Engineering},
  emajor={Computer Science and Technology},
  eauthor={Gao Bo},
  esupervisor={Assistant Professor Li Hongbo},
  % eassosupervisor={XXX},
  % 日期自动生成,若需指定按如下方式修改:
  % edate={December, 2005}
  %
  % 关键词用“英文逗号”分割
  ckeywords={信息融合, 深度学习, 识别, 神经网络},
  ekeywords={Information fusion, Deep learning, Recognition, Neural network}
}

% 定义中英文摘要和关键字
\begin{cabstract}
  物体的识别、标记,以及机器人的各类感知、操作的实现都基于对信息的处理、理解和利用。一方面,在实际操作中,单一模态的信息往往不能提供我们需要的精确度和全面性。这就为我们使用多模态信息,探究多模态信息的融合策略提供了动机。而另一方面,传感器技术的飞速发展,又为收集更高质量、更大规模的不同模态的信息提供了技术支持与可能性。所以本文就在此背景下进行多模态信息融合策略的研究,以便为机器人精细感知和精细操纵的实现提供技术基础。

  本文针对多方法多模态信息融合策略以及深度学习在多模态信息融合中的应用做出了一系列研究,具体的工作有:利用传统的方法来进行颜色和深度信息的提取和融合;提出一种决策层的带权投票信息融合策略;引入了深度学习方法;提出了一种将深度信息转换为合法图片的正规化方法;使用预训练模型提取多模态信息;提出了结合了分层匹配追踪、随机权值递归神经网络以及深度学习三种方法的综合多模态信息识别模型,提升了识别的准确率和稳定性。
\end{cabstract}

% 如果习惯关键字跟在摘要文字后面,可以用直接命令来设置,如下:
% \ckeywords{\TeX, \LaTeX, CJK, 模板, 论文}

\begin{eabstract}
  The processing, understanding and utilization of information serve as not only the basis of recognition and segmentation tasks, but also the key part in robotic fine perception and manipulation. On one hand, one single modal of information can not always provide the precision required by the practical case, therefore makes multi-modal information fusion a meaningful topic. On the other hand, the rapid development of sensor technology has made multi-modal information fusion viable by providing multi-modal datasets of larger scale and higher quality. In this article we present a multi-modal information fusion strategy combined with deep-learning methods, in order to provide technical backing for robotic perception and manipulation.
  
  We systematically propose a series of approches that can effectively fuse multi-modal information and multi-method features on different levels. Specifically, our research includes the following parts: using hierarchical matching pursuit and randomized CRNN to extract features from both RGB and depth cues; proposing a decision level fusion method based on SVM and weighed voting; taking adventage of deep-learning approach; proposing a depth normalization method that can apply different normalization scales to different parts of depth cues; using pretrained model to extract features; proposing a generalized fusion model based on hierarchical matching pursuit, randomized CRNN and deep-learning, which achieves high classification accuracy and stability.
\end{eabstract}

% \ekeywords{\TeX, \LaTeX, CJK, template, thesis}
